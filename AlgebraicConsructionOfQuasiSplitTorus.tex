\documentclass[12pt]{article}

%%%%%%%%%%%%%%%% 
\usepackage{fullpage}
\usepackage{appendix}
\usepackage{graphicx}
\usepackage{amsmath}
\usepackage{amsthm}
\usepackage[byname]{smartref}
\usepackage{hyperref} 
\usepackage{tocloft}
\usepackage{tikz}
\usepackage{xcolor}
\usepackage{listings}
%\usepackage{mathtools}
\usetikzlibrary{positioning}
\usepackage{tikz-qtree,tikz-qtree-compat}
\usetikzlibrary{trees}
\usetikzlibrary{arrows,positioning,automata,shadows,fit,shapes}
%\usepackage{mathtools}
\usepackage{float}
\usepackage{amssymb}
\usepackage{tikz-cd}
\usepackage{pagecolor,lipsum}
 %\pagecolor{lightgray}
\usepackage[english]{babel}
%\usepackage[table]{xcolor}
\usepackage{algorithmic}
\usepackage{algorithm}
\usepackage{authblk}
\usepackage{tikz-cd}
%\usepackage{chngcntr}
\renewcommand{\algorithmicrequire}{\textbf{Input:}}
\renewcommand{\algorithmicensure}{\textbf{Output:}}
%%%%%%%%%%%%%%%%%%%%%%%%%%%%
\usepackage{graphicx}
\usepackage{amsmath}
\usepackage{amsfonts}
\usepackage{mathrsfs}
\usepackage{bm}


%\hypersetup{
%	unicode = true,
%	colorlinks = true,
%	citecolor = blue,
%	filecolor = black,
%	linkcolor = black,
%	urlcolor = black,
%	pdfstartview = {FitH},
%}



\theoremstyle{plain}
\newtheorem{theorem}{Theorem}
\newtheorem{lemma}[theorem]{Lemma}
\newtheorem{corollary}[theorem]{Corollary}
\newtheorem{proposition}[theorem]{Proposition}
\newtheorem{definition}[theorem]{Definition}
\newtheorem{conjecture}[theorem]{Conjecture}
\newtheorem{example}[theorem]{Example}
\newtheorem*{remark}{Remark}
\newtheorem{note}[theorem]{Note}


\def\todo#1{(\textbf{todo:} #1)}
\newcommand{\N}{\ensuremath{\mathbb{N}}}
\newcommand{\Z}{\ensuremath{\mathbb{Z}}}
\newcommand{\Q}{\ensuremath{\mathbb{Q}}}
\newcommand{\C}{\ensuremath{\mathcal{C}}}
\newcommand{\Po}{\ensuremath{\mathcal{P}}}
\newcommand{\G}{G}
\newcommand{\glat}{$G$-lattice}
\newcommand{\tand}{\ensuremath{\,\,\, \text{and} \,\,\,}}
\newcommand{\exactseq}[1]{\ensuremath{0 \longrightarrow M_{#1} \longrightarrow L_{#1} \longrightarrow \Z \longrightarrow 0}}
\newcommand{\exactseqs}[1]{\ensuremath{0 \longrightarrow M_{#1} \longrightarrow L_{#1} \longrightarrow \Z^{-} \longrightarrow 0}}
\newcommand*{\QEDA}{\hfill\ensuremath{\blacksquare}}
\DeclareBoldMathCommand{\be}{e}


%%%%%%%%%%%%%%%%

\title{Algebraic Construction of Quasi Split Algebraic Tori}
\author{}

\begin{document}

\maketitle


\begin{abstract}

The main purpose of this work is to give a constructive proof for the No Name Lemma. Let $G$ be a finite group, $K$ be a field, $L$ be a permutation $G$-lattice with the standard basis and $K[L]$ be the group algebra of $L$ over $K$. The No Name Lemma asserts that the invariant field of the quotient field of $K[L]$, $K(L)^G$ is a purely transcendental extension of $K^G$. In other words, there exist $y_1, \ldots , y_n$ which are algebraically independent over $K^G$ such that $K(L)^G \cong K^G(y_1, \ldots , y_n)$. For a Galois extension $K/F$ with $G = \mathrm{Gal}(K/F)$ we have introduced $\mathcal{Y}=\lbrace y_1, \ldots, y_n \rbrace \subset K[L]^G$ with desired properties. Moreover, $\mathcal{Y}$ can be used to get a concrete description of $K[L]^G$. For a sign permutation $G$-lattice $L$, a more general argument is given so that we can concretely find a transcendence basis of $K(L)^G$ over $K^G$. Since the coordinate ring (resp. the rational function field) of an algebraic torus is given as invariant ring (resp. field), $K[L]^G$ (resp. $K(L)^G$) where $L$ is the character lattice of the algebraic torus, the given proof can be used to construct the group ring or rational function field of a quasi split algebraic torus.
\end{abstract}

\section{Introduction}



An algebraic $F$-torus $T$ is an algebraic group defined over a field
$F$ which {\em splits} over an algebraic closure $\bar F$ of $F$, that
is, which is isomorphic to a torus (a finite product of copies of the
multiplicative group $\mathbb{G}_m$) over $\bar{F}$. In general,
$\bar{F}$ is not the smallest field over which $T$ splits: it is known
that an algebraic $F$-torus $T$ splits over a finite Galois extension
of $F$. There is a unique minimal such extension, say $K$; if $G =
\mathrm{Gal}(K/F)$, then $G$ is called the splitting group of $T$. For
more details, see~\cite[Chapter 2]{Voskresenskii}.


A $G$-lattice $L$ is a free $\Z$-module $A$ of finite rank, together
with a group homomorphism $G \longrightarrow {\rm Aut}(L)$ (the group
of automorphisms of $L$). Given a module basis of $A$, any group
homomorphism $G \longrightarrow {\rm GL}(n,\Z)$, with $n={\rm
  rank}(A)$, gives such an an action.  If $K$ is a field, the group
algebra $K[L]$ of $L$ over $K$ is isomorphic to the $K$-algebra of
Laurent polynomials $K[x_1^{\pm 1},\dots,x_n^{\pm 1}]$, for some
indeterminates $x_1,\dots,x_n$.  If $K$ is equipped with an action of
$G$ (that is, a $G$-field), we can extend the action of $G$ on lattice
$L$ to an action on $K[L]$; the ring $K[L]^G$ of {\em multiplicative
  invariants} consists of those elements in $K[L]$ invariant under the
action of $G$. The fraction field $K(L)$ of $K[L]$ is isomorphic to
$K(x_1,\dots,x_n)$, and the subfield of invariants under the action of
$G$ is written $K(L)^G$.

It is known that there is a duality between the category of algebraic
tori with splitting group $G$ and $G$-lattices. For a given algebraic
torus $T$ with splitting group $G$, its character module
$\mathrm{Hom}(T,\mathbb{G}_m)$ is a $G$-lattice. Conversely, if $L$ is
a $G$-lattice, with $G={\rm Gal}(K/F)$ for some finite Galois
extension $K/F$, then $T=\mathrm{Spec}(K[L]^G)$ is an algebraic
$F$-torus with splitting group $G$, with coordinate ring $K[L]^G$ and
function field $K(L)^G$.

A $G$-lattice $L$ is called {\em permutation} (resp.\ {\em sign
  permutation}) if it has a $\Z$-basis which is permuted (resp.\ up to
sign changes) by $G$. The no-name lemma asserts that if $L$ is a
permutation $G$-lattice and $K$ is a $G$-field, then $K(L)^G$ is
rational over $K^G$~\cite[Chapter~9.4]{Lorenz}; in particular, with
$G={\rm Gal}(K/F)$, $K(L)^G$ is $F$-rational.

In this paper, we give a constructive proof of this fact, by
exhibiting a basis for such a field of invariants.  In concrete terms,
we start from a subgroup of $\mathrm{GL}(n,\Z)$ and describe the field
of functions of an associated torus.

\begin{definition}\label{Assumption}
  Let $G$ be a finite subgroup of $\mathrm{GL}(n,\Z)$.  The
  $G$-lattice $L_G$ corresponding to $G$ is the rank $n$ lattice
  generated by the standard basis $\langle \be_i : i = 1,
  \ldots, n \rangle_\Z$, where $\be_i$ is the column vector
  $[\delta_{i,j}, i=1,\dots,n]^T$, together with the action of $G$ given
  by left-multiplication on the $\be_i$'s.
\end{definition}

Suppose further that we are given an isomorphism $G \simeq
\mathrm{Gal}(K/F)$, for some finite Galois extension $K/F$. Then,
through the identification $K[x_1^{\pm 1}, \ldots , x_n^{\pm
    1}]\simeq K[L]$, $K[x_1^{\pm 1}, \ldots , x_n^{\pm 1}]$ is
equipped with the $G$-action given by
\begin{itemize}
\item $G$ acting as Galois group on $K$
\item for $g$ in $G$ and $i=1,\dots,n$, $g(x_i) = \prod_{j=1}^{j=n}
  x_j^{g_{i,j}}$, where $g_{i,j}$ is the $(i,j)$th -entry of $g$
 (so that we also have $g(\be_i) =
  \sum_{j=1}^{n} a_{i,j}\be_j$).
\end{itemize}
If we let $T_G$ be the algebraic torus corresponding to $L_G$, then
$T_G$ is an algebraic $F$-torus which splits over $K$, with character
lattice $L_G$ and function field $K(x_1,\dots,x_n)^G$.

Conjugate subgroups of $\mathrm{GL}(n,\Z)$ correspond to isomorphic
lattices, and isomorphic algebraic tori; in particular, for $G$ a
finite subgroup of $\mathrm{GL}(n,\mathbb{Z})$, $L_G$ is a (signed)
permutation lattice if and only if $G$ is conjugate to a group of
(signed) permutation matrices.  Computationally, we do not have an
efficient algorithm at hand to decide whether a given lattice is
(signed) permutation. Hence, in our main results, we will assume that
$G$ is a subgroup of the group $\mathbb{S}_n$ of permutation matrices
of size $n$, or more generally of the group $\mathbb{B}_n$ of signed
permutation matrices of size $n$.  In such a case, for $i$ in
$\{1,\dots,n\}$ and $g$ in $G$, $g(\be_i)=\pm \be_j$ for some index
$j$ in $\{1,\dots,n\}$ (all signs being $+1$ if $G$ is a subgroup of
$\mathbb{S}_n$), and the action of $g \in G$ on $x_i$ is given
by $$g(x_i) = \begin{cases} x_j & \text{if~} g(\be_i) = \be_j
  \\ x^{-1}_j & \text{if~} g(\be_i) = -\be_j. \end{cases}$$

For such groups $G$, the $F$-rationality of the torus $T(L_G)$ means
that for $K(x_1,\dots,x_n)$, endowed with the $G$-action we just
described, there exist $y_1,\dots,y_n$ such that
$K(x_1,\dots,x_n)^G=F(y_1,\dots,y_n)$. However, the proofs of the
no-name lemma we are aware of~\cite{xxx} are non constructive. 
The goal of this paper is to exhibit such $y_i$'s; we state several 
such results.


The proof is based on a generalization of the Moore
determinant~\cite[Section 1.3]{Goss}.




%% There exists a homomorphism $\rho: \mathbb{B}_n \longrightarrow
%% \mathfrak{S}_n$, which to a signed permutation matrix $g$ associates
%% the permutation $\rho(g) = \rho_g$ obtained by forgetting the signs in
%% $g$. 

%%  In addition, for $g$ in $\mathbb{B}_n$, we can define $s_g:
%% \{1,\dots,n\} \to \{-1,1\}$ (such that $s_g s_h =s_{gh}$ for all such
%% $g,h$), such that the action of $g \in \mathbb{B}_n$ on an element of
%% the standard basis $\mathbf{e}_i$ is given by $g(\mathbf{e}_i) =
%% s_g(i) \mathbf{e}_{\rho_g^{-1}(i)}$. Similarly, for the action on
%% $K[x_1^{\pm 1}, \ldots , x_n^{\pm 1}]$, we have $g(x_i) =
%% x_{\rho_g^{-1}(i)}^{s_g(i)}.$






 

%% When $L_G$ is a permutation lattice, by the no-name lemma,
%% $K(T_{G,F})$ is a rational extension of $F$, i.e. there exist
%% $y_1,\ldots,y_n \in K[x_1, \ldots , x_n]$ such that $$K(x_1, \ldots,
%% x_n)^G = F(y_1,\ldots,y_n)$$ and $y_1, \ldots, y_n$ are algebraically
%% independent over $F$.  The existence of $y_i$'s is related to the
%% existence of a permutation basis for $L_G$. Hence having such a basis,
%% enables us to construct the transcendence basis we are looking for.


\paragraph{Theorem}
Assume $G \leq \mathrm{GL}(n,\Z)$ and the corresponding $G$-lattice,
$L_G$ (as in Definition \ref{Assumption}), is sign
permutation. Suppose that $G$ acts transitively (up to sign) on
$L_G$. Let $K/F$ be a finite Galois extension with Galois group
$G$. Let $\alpha \in K$ be a normal element for the Galois extension
$K/F$. Then
$$K(L_G)^G = K(x_1,\ldots, x_n)^G = F(y_1, \ldots, y_{n}),$$ where $S
= \sum_{g \in G} g \in \Z[G]$ and $y_i = S(\alpha (1+x_i)^{-1})$ for $
i = 1, \ldots, n$.

An algebraic torus is defined without using an ideal of a polynomial
ring. Even the function field (resp. coordinate ring) of a general
algebraic torus is defined as the field of invariants of a field
(resp. ring of) under the action of some finite group. Considering the
fact that these invariants are multiplicative, it is not easy to find
these invariants in general. To the best of the authors knowledge,
there are only a few algorithms for finding the multiplicative
invariants (see \cite{Kemper} and \cite{Lorenz}). The results
presented in this chapter allow us to find multiplicative invariants
in the particular cases where the lattice is permutation or sign
permutation.

We have already seen the duality between algebraic tori and
lattices. For a given $G \leq \mathrm{GL}(n,\Z)$ although we know
$K(T_{G,F}) \cong K(x_1, \ldots, x_n)^G$ and $K[T_{G,F}] \cong
K[x^{\pm 1}_1, \ldots , x^{\pm 1}_n]^G$ , it is given as a field (or
ring) of multiplicative invariants, and we do not have a generating
set for them. We are interested in finding the multiplicative
invariants in a concrete way.  There are many algorithms for finding
the invariant rings for polynomial invariants. We invite the reader to
consult \cite{Kemper2} and \cite{Sturmfels}. However, for
multiplicative invariants the algorithmic side is not explored to some
extent \cite{Kemper}, \cite{Renault}.

We present results concerning the construction of algebraic function
field and coordinate ring of quasi split tori. The first section is
devoted to a brief discussion of the needed material. In the second
section, a constructive proof of the No Name Lemma is presented. This
can be used to find an explicit transcendence basis of the rational
function field of a quasi split torus. The final section presents a
similar result for finding the function field of an algebraic torus
with sign permutation lattice.

\begin{definition}
  Assume $K/F$ is a finite Galois extension and $G = \lbrace \sigma_1,
  \ldots , \sigma_n \rbrace$ is the Galois group of $K/F$. An element
  $\alpha \in K$ is called normal if $B = \lbrace \sigma_1(\alpha),
  \ldots , \sigma_n(\alpha) \rbrace$ is an $F$-basis for $K$, in which
  case we call $B$ a normal basis of $K$ over $F$.
\end{definition} 

Any finite Galois extension admits a normal element~\cite[Theorem
  6.13.1]{Lang}; there exist algorithms to construct one, in
characteristic zero~\cite{Girstmair} or in positive
characteristic~\cite{Giesbrecht,Poli}.


%%%%%%%%%%%%%%%%%%%%%%%%%%%%%%%%%%%%%%%%%%%%%%%%%%%%%%%%%%%%
%%%%%%%%%%%%%%%%%%%%%%%%%%%%%%%%%%%%%%%%%%%%%%%%%%%%%%%%%%%%
%%%%%%%%%%%%%%%%%%%%%%%%%%%%%%%%%%%%%%%%%%%%%%%%%%%%%%%%%%%%


\section{The case of the whole symmetric group}
 
\todo{Here, I am doing as if the action is on the right.}

\begin{theorem}\label{specialcase}
  Let $K/F$ be a finite Galois extension with Galois group
  $\mathfrak{S}_n \simeq \mathbb{S}_n$ where $K$ is the splitting
  field of an irreducible polynomial $ f \in F[x]$. Assume $R =
  \lbrace r_1, \ldots, r_n \rbrace$ is the set of all roots of $f$ in
  $K$, numbered such that for $g \in \mathfrak{S}_n$, $g(r_i) =
  r_{g(i)}$, and that $2,\dots,n-2,n$ are units in $F$. Define
  \begin{align*}
    y_1 &=x_1+ x_2+ \cdots + x_n\\
    y_i &= \sum_{g\in \mathbb{S}_n} g(r_i x_i) \qquad i = 2, \ldots, n.
  \end{align*}
  Then we have $K(x_1, \ldots, x_n)^{\mathbb{S}_n} = F(y_1, \ldots, y_n).$
  %% Moreover the
  %% coefficient of $x_k$ in $y_i$ for $1 \leq i \leq n-1$ is given by:$$
  %% c_k = \begin{cases} (n-1)!r_i & k=i \\ (n-2)!(A-r_i) & k\neq
  %%   i \end{cases}$$
\end{theorem}
\begin{proof}
    The elements $(y_1,\dots,y_n)$ are invariant under the action of
    $\mathbb{S}_n$. We will show below that they are $K$-linearly independent;
    this will prove that $K(x_1,\dots,x_n)^{\mathbb{S}_n}=F(y_1,\dots,y_n)$, since
    then $K(x_1,\dots,x_n)=K(y_1,\dots,y_n)$, and
    $K(y_1,\dots,y_n)^{\mathbb{S}_n}=K^{\mathbb{S}_n}(y_1,\dots,y_n)=F(y_1,\dots,y_n)$.

    For $i=2,\dots,n$, we have
    $$y_i = \sum_{j=1}^n c_j x_j,\quad\text{with}\quad c_j = \sum_{g\in G_{i,j}} g(r_i)=\sum_{g\in G_{i,j}} r_{\rho_g(i)}.$$
    If $i=j$, then $G_{i,i}$ is the stabilizer ${\rm Stab}_{\mathfrak{S}_n}(i)$,
    so it has cardinality $(n-1)!$, and $g(r_i)=r_i$ for all $g$ in it.
    For $i \ne j$, $g \in G_{i,j}$ means that $g(x_i)=x_j$, that is,
    $\rho_{g}^{-1}(x_i)=x_j$, so that $\rho_g$ maps $\{1,\dots,n\}-\{j\}$
    bijectively to $\{1,\dots,n\}-\{i\}$. For any element $r_k$ in $\{1,\dots,n\}-\{i\}$,
    there are $(n-2)!$ permutations $g$ in $G_{i,j}$ with $\rho_g(i)=k$, so that
    $$c_j = (n-2)! (r_1 +\cdots +r_n -r_i) = (n-2)!(A-r_i),$$
    where $A\in F$ is the coefficient of degree $n-1$ in $f$. So we can rewrite
    \begin{align*}
    y_i &= (n-1)! r_i x_i + (n-2)! (A-r_i)(x_1 + \cdots +x_n -x_i)\\      
&= (n-2)! \big ( (n-1) r_i x_i + (A-r_i)(y_1-x_i) \big )\\
    &= (n-2)! \big( (A-r_i) y_1 + ( nr_i -A)x_i \big ).
    \end{align*}
    Remark that for all $i$, $nr_i-A$ is non-zero, since otherwise 
    $r_i=A/n$ would imply that $r_i$ is in $F$. Taking into account the fact that 
    $y_1=x_1 +\cdots +x_n$, this shows that 
    the family $(y_1,\dots,y_n)$ is $K$-linearly independent.
\end{proof}

The lattice $L_{\mathbb{S}_n}$ in the previous theorem is isomorphic
to the permutation $\mathfrak{S}_n$-lattice
$\Z[\mathfrak{S}_n/\mathfrak{S}_{n-1}]$. For a geometric description of
the corresponding algebraic torus, see Examples 18 and 19
in~\cite{Voskresenskii}.

\begin{example}	
Suppose that $n=2$, so that $G=\mathbb{S}_2 =\{ {\rm Id}, \sigma \}$,
with ${\rm Id}$ the identity matrix of size 2 and $\langle \sigma
= \begin{smallmatrix} 0&1\\ 1&0
\end{smallmatrix}\rangle$. We  take $F = \Q$ and $K = \Q(i)$ as our Galois 
extension; then $\sigma(i) = -i$. Now $L_{\mathbb{S}_2} = \langle \be_1,\be_2
\rangle_{\Z}$ is a permutation ${\mathbb{S}_2}$-lattice with $K(L_{\mathbb{S}_2}) = K(x_1,x_2)$,
and we obtain that $$y_1 = x_1+x_2\quad \textrm{and}\quad y_2 = ix_1
-ix_2 $$ are such that $$K(L_{\mathbb{S}_2})^{\mathbb{S}_2} = K(x_1,x_2)^{\mathbb{S}_2} = \Q(y_1,y_2).$$
\end{example}

\begin{example}
Take $n=3$, so that $G=\mathbb{S}_3 \subset \mathrm{GL}(3,\Z)$ is
generated by $$\sigma = \begin{bmatrix} 0 & 0 &1\\ 1 & 0 &0\\ 0 & 1 &
  0
\end{bmatrix}\,\,\, \text{and} \,\,\, \tau = \begin{bmatrix}
0 & 1 &0\\
1 & 0 &0\\
0 & 0 & 1
\end{bmatrix}.$$
We take for $K$ the splitting field of $x^3-2$, that is, $K=\Q(\rho,
\sqrt[3]{2})$, where $\rho$ is a primitive third root of unity.

The roots of $x^3-2$ are $\sqrt[3]{2},\rho \sqrt[3]{2}, \rho^2 \sqrt[3]{2}$. The Galois group of the extension is $\mathrm{S}_3$ with the action $$\sigma = \begin{cases} \sqrt[3]{2}\longrightarrow \rho \sqrt[3]{2} \\ \rho \longrightarrow \rho \end{cases}\,\,\, \,\,\,\tau = \begin{cases} \sqrt[3]{2}\longrightarrow \sqrt[3]{2} \\ \rho \longrightarrow \rho^2 \end{cases}$$

Let $r_1 = \rho \sqrt[3]{2}, r_2 = \rho^2 \sqrt[3]{2}$ and $r_3 = \sqrt[3]{2}$. One can verify that for $g \in G$, $g(r_i) = r_{g(i)}$ and then we get
 $$S= 1+\sigma +\sigma^2 +\tau +\sigma \tau +\tau \sigma $$
 $$y_0 = S(x_1+ x_2 + x_3)= 6(x_1+x_2+x_3)$$
 $$ y_1 = S(\rho \sqrt[3]{2} x_1) = 2\rho\sqrt[3]{2}x_1+(\sqrt[3]{2}+\rho^2 \sqrt[3]{2})x_2 + (\sqrt[3]{2}+\rho^2 \sqrt[3]{2})x_3 $$
 $$ y_2 = S(\rho^2\sqrt[3]{2}x_2) = (\sqrt[3]{2}+\rho \sqrt[3]{2})x_1 + 2\rho^2\sqrt[3]{2}x_2 + (\sqrt[3]{2}+\rho \sqrt[3]{2})x_3$$
 \end{example}

\section{Another construction for subgroups of $\mathbb{S}_n$}

In Theorem \ref{specialcase}, we worked with $G=\mathbb{S}_n$ and
assumed the roots of defining polynomial for our Galois extension are
given.  In this section, we give an alternative approach that works
for subgroups of $\mathbb{S}_n$, but we will now assume that a normal
element of the field extension is given.

Let thus $G$ be a finite subgroup of $\mathbb{S}_n$, let $K/F$ be a
finite Galois extension with Galois group $G$ and let $M$ be in
$M_{m,n}(K)$, with $m \leq n$. For such a matrix, and $j=1,\dots,n$,
its $j$th column is written $M_j=\left[\mu_{1,j} \,\,\, \mu_{2,j}
  \,\,\, \cdots \,\,\, \mu_{m,j}\right]^T$. 

The group $G$ acts on (column) vectors entrywise.

For $g \in G$ and a column
of $M$, written as $M_j = \left[\mu_{1,j} \,\,\, \mu_{2,j} \,\,\,
  \cdots \,\,\, \mu_{m,j}\right]^T$, we define $g(M_j) =
\left[g(\mu_{1,j}) \,\,\, g(\mu_{2,j}) \,\,\, \cdots \,\,\,
  g(\mu_{m,j})\right]^T$.

%% We say that $G$ permutes the columns of $M$ transitively if there
%% exists a homomorphism $\rho: G \longrightarrow \mathrm{S}_n$, $\rho(g)
%% = \rho_g$ such that $g(M_i)= M_{\rho_g(i)}$ for some $s_g
%% \in \lbrace 0, 1\rbrace$ for all $i = 1, \ldots n$ and for each $1
%% \leq i \neq j \leq n$, there exists $g \in G$ such that $\rho_g(i) =
%% j$. Note that the action of $G$ on the columns of $M$ is not required
%% to be faithful.
\begin{lemma}\label{signdet}
   Let $K/F$ be a finite Galois extension with finite Galois group
   $G$. Let $M \in M_{mn}(K)$, $m \leq n$ and assume that $G$ permutes
   the columns of $M$ transitively up to sign. Assume also that the
   entries of $M_1$ are $F$-linearly independent. Then the rows of $M$
   are $K$-linearly independent.
\end{lemma}
\begin{proof}
The proof is by induction on $m$. If $m = 1$, we need only to show
that the unique row of $M$ is non-zero. This is true since if $M_1 =
[v_1]$, $v_1$ is $F$ linearly independent and so non-zero. Since $v_1$
is the first entry in the only row of $M$, we are done.

Now assume that $m >1$. To show that the rows of $M$ are linearly
independent over $K$, it is equivalent to show that the null space of
$M^T$ is trivial. We will show this by contradiction. Assume that
there exists $\textbf{0} \neq \textbf{x} \in N(M^T) \subseteq K^m$. So
$M^T\textbf{x} = 0$. There exists some $x_k \neq 0$. Let $\textbf{y}=
\frac{1}{x_k}\textbf{x} \in K^m$. Then $y_k = 1$ and $\textbf{y}\in
N(M^T)$, so $M^T \textbf{y}=0$. The $i$th component is $M^T_i
\textbf{y}= 0 $, $i = 1, \ldots, n$. For each $g \in G$, we get
$g(M^T_i \textbf{y}) = g(M_i)^Tg(\textbf{y}) = \pm
M^T_{\rho_g(i)}g(\textbf{y}) = 0$ for all $i = 1, \ldots , n$ and so
$M^T_jg(\textbf{y}) = 0$ for all $j = 1, \ldots , n$, which shows that
$g(\textbf{y}) \in N(M^T)$. So $g(\textbf{y}) -\textbf{y} \in
N(M^T)$. By assumption, the $k$th component of $g(\textbf{y}) -
\textbf{y} $ is $0$, and so $g(\hat{\textbf{y}}) - \hat{\textbf{y}}\in
N(\hat{M}^T)$ where $\hat{\textbf{y}} \in K^{m-1}$ is the vector
$\textbf{y}$ with the $k$th component removed and $\hat{M} \in
M_{m-1,n}(K)$ is $M$ with row $k$ removed. Note that $\hat{M}$ has
columns $\hat{M}_i$, $i = 1, \ldots, n$. Since $M_1$ has entries which
are $F$-linearly independent, so does $\hat{M}_1$. Since the columns
of $M$ are permuted transitively up to sign changes by the action of
$G$, so the columns of $\hat{M}$ are similarly permuted transitively
up to sign changes. Since the inductive hypothesis applies to
$\hat{M}$, we see that the rows of $\hat{M}$ are $K$-linearly
independent, or equivalently $N(\hat{M}^T)$ is trivial. Since
$g(\hat{\textbf{y}}) -\hat{\textbf{y}} \in N(\hat{M}^T) = \lbrace 0
\rbrace$ for all $ g \in G$, we see that $\hat{\textbf{y}}\in F^{m-1}$
and so $\textbf{y} \in F^m$. But then $M^T_1 \textbf{y} = 0 $ is
equivalent to $\sum^m_{k = 1}v_ky_k = 0$ which is a non-trivial
$F$-dependence relation for the entries of the first column of $M$. By
contradiction, the rows of $M$ must be $K$-linearly independent and so
rank$(M) = m$.
 \end{proof} 


\begin{theorem}\label{theo:1}
  Let $G$ be a subgroup of $\mathbb{S}_n$, let $K/F$ be a finite
  Galois extension with Galois group $G$, and let $\alpha \in K$ be a
  normal element for $K/F$. Then $ K(x_1,\ldots ,
  x_n)^G=F(y_1,\dots,y_n)$, with
  $$y_i =\sum_{g \in G} g(\alpha x_i) \quad i =
  1,\dots,n.$$
\end{theorem}
\begin{proof}
  We prove the result under the additional assumption that $G$ acts
  transitively on $L_G$.  The elements $(y_1,\dots,y_n)$, with $y_i =
  \sum_{g \in G} g(\alpha x_i)$ as defined above, are invariant under
  the action of $G$. We will show below that they are $K$-linearly
  independent; this will prove that
  $K(x_1,\dots,x_n)^G=F(y_1,\dots,y_n)$, since then
  $K(x_1,\dots,x_n)=K(y_1,\dots,y_n)$, and
  $K(y_1,\dots,y_n)^G=K^G(y_1,\dots,y_n)=F(y_1,\dots,y_n)$.

  For $i,j$ in $\{1,\dots,n\}$, let $G_{i,j}= \lbrace g \in G: g(x_i)
  = x_j \rbrace$, so that we can rewrite $y_i$ as $$y_i = \sum^n_{j
    =1}\sum_{g \in G_{i,j}}g(\alpha)x_j, \,\,\, i = 1, \ldots, n.$$
  Since the action of $G$ on $L_G$ is transitive, $G_{i,j}$ is
  non-empty for every $1 \leq i,j \leq n$.  Take such indices $i,j$,
  and fix some $g_{i,j}$ in $G_{i,j}$. If $g \in G_{i,j}$, then
  $g^{-1}_{i,j}g(x_i) = x_i$ shows that $g$ is in
  $g_{i,j}\mathrm{Stab}_G(x_i)$. Since we also have
  $g_{i,j}\mathrm{Stab}_G(x_i) \subseteq G_{i,j}$, we see that
  $G_{i,j} = g_{i,j}\mathrm{Stab}_G(x_i)$.


  We now show that the matrix $M$ with $i$th row the coordinate vector
  of $y_i$ with respect to the $K$-basis $\lbrace x_1, \ldots x_n
  \rbrace$ is invertible. The matrix $M$ has entries $m_{i,j} =
  \sum_{g \in G_{i,j}}g(\alpha)$, $i,j = 1, \ldots, n$. We will apply
  Lemma~\ref{signdet} to show that $M$ has $K$-linearly independent
  rows.

  We need to check the hypothesis of the lemma are satisfied. First,
  let $\rho: G \longrightarrow \mathfrak{S}_n$, $\rho(g) = \rho_g$ be
  the group homomorphism that corresponds to the action of $G$ on the
  $\lbrace x_1, \ldots , x_n \rbrace$, so that $\rho_g(i) = j$ if and
  only if $g(x_i) = x_j$ for all $1 \leq i,j\leq n$. We will show that
  the columns of $M$ are permuted by the action of $G$. Let thus $h$ be in
  $G$. Note that for $g$ in $G_{i,j}$, $hg$ is in $G_{i,\rho_h(j)}$;
  since  $G_{i,\rho_h(j)}$ and $G_{i,j}$ have the same cardinality,
  equal to $|\mathrm{Stab}_G(x_i)|$, we get
  $$h(m_{i,j}) = \sum_{g \in G_{i,j}}hg(\alpha) = \sum_{\sigma \in
    G_{i, \rho_{h}(j)}}\sigma (\alpha) = m_{i,\rho_{h}}(j).$$ This
  shows that $h(M_j) = M_{\rho_h(j)}$ for all $j = 1, \ldots, n$, so
  that $G$ permutes the columns of $M$. Since the action 
  of $G$ on $L_G$ is transitive, the action of $G$ on the columns 
  of $M$ is transitive as well.
  
  Finally, the first column $M_1$ has entries $[ \sum_{g \in
      G_{i,1}}g(\alpha),\ i = 1, \ldots,n ].$ Since $\alpha$ is a
  normal element of the Galois extension $K/F$ with Galois group $G$,
  the set $\lbrace g(\alpha): g \in G \rbrace$ is $F$-linearly
  independent. Since $G = \sqcup^n_{i =1}G_{i1}$ is a disjoint union,
  the set
  $$\left\lbrace \sum_{g \in G_{i,1}}g(\alpha), i = 1, \ldots, n \right\rbrace$$ is
  $F$-linearly independent as well.
  
  So Lemma~\ref{signdet} applies, and we conclude that $y_1, \ldots,
  y_n$ are $K$-linearly independent, as claimed.

  \medskip 
  
  We can now give the proof of our property in the general case.
  Let $\be_1, \ldots , \be_n$ be the standard basis of
  $L_G$, and let $\{\be_{j_k} \mid k = 1, \ldots, r\}$ and
  correspondingly $\{x_{j_k} \mid k = 1, \ldots, r\}$ be a complete set
  of $G$-orbit representatives among the basis vectors, and the
  indeterminates $x_1, \ldots, x_n$ respectively. Then $L_k = \oplus_{
    \be_i \in G \be_{j_{k}} } \Z \be_i$ is a
  transitive permutation $G$-lattice for each $k = 1, \ldots, r$, and
  $K(L_k) = K(x_i \mid x_i \in Gx_{j_k})$.
  
  The lattice $L_G = \oplus^r_{k =1} P_k$ is a direct sum of transitive
  permutation $G$-lattices, and similarly $K(x_1,\dots,x_n)^G$ is the
  compositum of the fields $K(P_k)$, $k=1,\dots,r$. Thus, using the 
  result established in the transitive case, we obtain
  $K(x_1,\dots,x_n)^G=F(y_i \mid x_i \in Gx_{j_k}, k=1,\dots,r)$, where
  for all $k$ and for $x_i \in Gx_{j_k}$, we have $y_i=\sum_{g \in G}
  g(\alpha x_i)$.
\end{proof}

%% \begin{corollary}\label{ConstructiveNoNameLemma}
%%   Let all assumptions be as in the previous theorem, except that
%%   we do not assume the action of $G$ to be transitive.
%%   Then $ K(x_1,\ldots , x_n)^G=F(y_1,\dots,y_n)$, with
%%   $$y_i =\sum_{g \in G} g(\alpha x_i) \quad i =
%%   1,\dots,n.$$
 %% where $y_i = S(\alpha x_i) = \sum_{x_j \in Gx_i}\sum_{g \in
 %%    G_{ij}}g(\alpha)x_j$, $i = 1, \ldots , n$ where $S = \sum_{g\in G}
 %%  g \in \Z[G]$ and $Gx_i = \lbrace gx_i : g \in G \rbrace$ and $G_{ij}
 %%  = \lbrace g \in G : g(x_i) = x_j \rbrace$. 
%% \end{corollary}
%% \begin{proof}
%% %% $y_i = \sum_{x_j \in Gx_{j_k}}\sum_{g \in
%%   %% G_{ij}}g(\alpha)x_j$
%%  Since $x_i \in Gx_{j_k}$, we see that $x_j \in
%% Gx_{j_k}$ if and only if $x_j \in Gx_i$ so we may express
%% $$y_i = \sum_{x_j \in Gx_i}\sum_{g \in G_{ij}}g(\alpha)x_j$$ 
%% (Note also that in fact, $G_{ij}$ is non-empty if and only if $x_j \in Gx_i$, so we could even write 
%% $$y_i = \sum^n_{j=1}\sum_{g \in G_{ij}}g(\alpha)x_j$$
%% as before). At any rate $K(P)^G = F(y_i : x_i \in Gx_{j_k}, k = 1, \ldots,r) = F(y_1, \ldots,y_n)$ as required.
%% \end{proof}
\begin{example}
Let $K$ be the splitting field of $x^4-2$ over $\Q$. Then $\mathrm{Gal}(K/\Q) \cong D_8$, $K = \Q(\sqrt[4]{2},i)$ and $\lbrace 1, \theta, \theta^2, \theta^3, i, i\theta, i\theta^2, i\theta^3\rbrace$ where  $\theta = \sqrt[4]{2}$ is a $\Q$-basis for $K$. Moreover, let $G\leq \mathrm{GL}(4,\Z)$ be generated by 
$$
r = \begin{bmatrix}
0&0&0&1\\
1&0&0&0\\
0&1&0&0\\
0&0&1&0
\end{bmatrix} \,\,\,\, \text{and}\,\,\,\,
s = \begin{bmatrix}
1&0&0&0\\
0&0&0&1\\
0&0&1&0\\
0&1&0&0
\end{bmatrix}.
$$
One can verify that $G \cong D_8$. The action of $r$ and $s$ on the basis of $K$ is given by
$$
\begin{matrix}
r(i) = i &r(\theta) = i \theta \\
s(i) = -i & s(\theta) = \theta
\end{matrix}
$$
Now we define $$\alpha = 1+ \theta + \theta^2 + \theta^3 +i + i\theta + i\theta^2 + i\theta^3$$
and claim that $\alpha$ is a normal element in $K$. In fact, 
%$$\alpha_2 =  r(\alpha) = 1- \theta - \theta^2 + \theta^3 +i + i\theta - i\theta^2 - i\theta^3$$
%$$\alpha_3 = r^2(\alpha) = 1- \theta + \theta^2 - \theta^3 +i - i\theta + i\theta^2 - i\theta^3$$
%$$\alpha_4 = r^3(\alpha) = 1+ \theta - \theta^2 - \theta^3 +i - i\theta - i\theta^2 + i\theta^3$$
%$$\alpha_5 = s(\alpha) = 1+ \theta + \theta^2 + \theta^3 -i - i\theta - i\theta^2 - i\theta^3$$
%$$\alpha_6 = rs(\alpha) = 1+ \theta - \theta^2 - \theta^3 -i + i\theta+ i\theta^2 - i\theta^3$$
%$$\alpha_7 = r^2s(\alpha) = 1- \theta + \theta^2 - \theta^3 -i + i\theta - i\theta^2 + i\theta^3$$
%$$\alpha_8 = r^3s(\alpha) = 1- \theta - \theta^2 + \theta^3 -i - i\theta + i\theta^2 + i\theta^3$$
Define $y_i = S(\alpha x_i)$ where $$S = 1+r+r^2+r^3+s+sr+sr^2+sr^3 \in \Z[D_8].$$ Finally, $$K(x_1, x_2, x_3,x_4) = F(y_1,y_2,y_3,y_4).$$
It is also worth presenting the coordinate matrix of the $y_i$'s, as a concrete example of Lemma \ref{signdet}. 
%In order to do so, we need to know the action of $G$ on the $x_i$'s.
%\begin{table}[H]
%\centering
%\begin{tabular}{l|llllllll} 
% & $1$ & $r$ & $r^2$ & $r^3$ & $s$ & $sr$ & $sr^2$ & $sr^3$\\
% \hline
% $x_1$  & $x_1$ & $x_4$ & $x_3$ & $x_2$ & $x_1$ & $x_2$ & $x_3$ & $x_4$\\
%$x_2$ & $x_2$ & $x_1$ & $x_4$ & $x_3$ & $x_4$ & $x_1$ & $x_2$ & $x_3$\\
%$x_3$ & $x_3$ & $x_2$ & $x_1$ & $x_4$ & $x_3$ & $x_4$ & $x_1$ & $x_2$\\
%$x_4$ & $x_4$ & $x_3$ & $x_2$ &  $x_3$& $x_2$ & $x_3$ & $x_4$ & $x_1$\\
%\end{tabular}
%\end{table}
%\medskip

From the above table, one can form the matrix 
$$
M= \begin{bmatrix}
(1+s)(\alpha) & (r^3+sr)(\alpha) & (r^2+sr^2)(\alpha) & (r+sr^3)(\alpha)\\
(r+sr)(\alpha) & (1+sr^2)(\alpha) & (r^3+sr^3)(\alpha) & (r^2+s)(\alpha)\\
(r^2+sr^2)(\alpha) & (r+sr^3)(\alpha) & (1+s)(\alpha) & (r^3+sr)(\alpha)\\
(r^3+sr^3)(\alpha) & (r^2+s)(\alpha) & (r+sr)(\alpha) & (1+sr^^2)(\alpha)
\end{bmatrix}.
$$
The action of $r$ and $s$ on the columns is
\begin{table}[H]
\centering
\begin{tabular}{l|llllllll} 
 & $r$ & $s$ \\
 \hline
 $M_1$  & $M_4$ & $M_1$ \\
$M_2$ & $M_1$ & $M_4$ \\
$M_3$ & $M_2$ & $M_2$ \\
$M_4$ & $M_3$ & $M_3$ \\
\end{tabular}
\end{table}

\end{example} 

As has been mentioned above, we can apply Lemma \ref{signdet} in order to compute the coordinate ring of an algebraic torus. 
\begin{theorem}
With the assumptions of Theorem \ref{ConstructiveNoNameLemma} $$K[L]^G \cong K[x^{\pm 1}_1, x^{\pm 1}_2, \ldots , x^{\pm 1}_n]^G = F[y_1, \ldots , y_n]_{x_1\cdots x_n}$$ where $ y_i$ is given by $$S = \sum_{\sigma \in G} \sigma \in \Z[G]$$
$$y_i = S(\alpha x_i)  \,\,\,\,\, \text{for} \,\,\, 1\leq i \leq n.$$
\end{theorem}
\begin{proof}
It is known that $K(L)$ is isomorphic to a Laurent polynomial ring. Also $$K[x^{\pm 1}_1, x^{\pm 1}_2, \ldots , x^{\pm 1}_n] = K[x_1, \ldots , x_n]_{x_1\cdots x_n}.$$ We are interested in $K[L]^G \cong \left( K[x^{\pm 1}_1, x^{\pm 1}_2, \ldots , x^{\pm 1}_n] \right)^G = \left(K[x_1, \ldots , x_n]_{x_1\cdots x_n} \right) ^G.$ By the proof of Theorem \ref{ConstructiveNoNameLemma} we can see $K[x_1, \ldots , x_n] = K[y_1, \ldots , y_n]$. 
%and for each $i$, there exist $f_i(y_1,\ldots, y_n) \in K[y_1, \ldots , y_n]$ such that $x_i = f_i(y_1,\ldots, y_n)$. This implies  $$K[x_1, \ldots , x_n]_{x_1\cdots x_n} =  K[y_1, \ldots , y_n]_{h(y_1, \ldots, y_n)}$$ where $h(y_1,\cdots, y_n) = \prod_1^n f_i(y_1 , \ldots, y_n)= x_1 \cdots x_n.$ 
On the other hand since $G$ permutes the $x_i$'s, $x_1\cdots x_n$ is invariant under the action of $G$, we can conclude $$\left( K[x_1, \ldots , x_n]_{x_1\cdots x_n}\right)^G =  \left( K[y_1, \ldots , y_n]_{x_1\cdots x_n} \right)^G$$$$ = K^G [y_1, \ldots , y_n]_{x_1\cdots x_n} =  F[y_1, \ldots , y_n]_{x_1\cdots x_n}.$$ 
\end{proof}

\section{Algebraic tori with sign permutation character lattice}


\begin{theorem}\label{nonamesign}
  Let $G$ be a subgroup of $\mathbb{B}_n$, let $K/F$ be a finite
  Galois extension with Galois group $G$, and let $\alpha \in K$ be a
  normal element for $K/F$. Then
  $$K(L_G)^G = K(x_1,\ldots, x_n)^G = F(y_1, \ldots, y_{n}),$$ 
  where $y_i = \sum_{g \in G} g\left (\frac{\alpha}{1+x_i}\right)$ for $i$ in $\{1, \ldots, n\}$.
\end{theorem}
\begin{proof}
  The proof follows that of Theorem~\ref{theo:1}; the only difference
  is in the description of the coordinate matrix $M$.
  As in Theorem~\ref{theo:1}, we first prove the result under the
  extra assumption that that $G$ acts transitively up to sign on
  $L_G$.


  For $i$ in $\{1,\dots,n\}$, define $z_i = (1+x_i)^{-1} $. Now for $g
  \in G$, $$g(z_i) = \begin{cases} z_j & \text{if} \,\,\, g(x_i) = x_j
    \\ 1-z_j & \text{if} \,\,\, g(x_i) = x_j^{-1},
  \end{cases}$$ 
  and $K(x_1, \ldots , x_n) = K(z_1, \ldots, z_n).$ The elements $y_i$
  can be rewritten as $y_i = \sum_{g \in G} g ({\alpha}z_i)$, for $i$
  in $\{1, \ldots, n\}$; as before, in order to prove that
  $K(z_1,\ldots, z_n)^G = F(y_1, \ldots, y_{n})$, it is enough to
  prove that $y_1,\dots,y_n$ are $K$-linearly independent.

  We actually prove that $(1,y_1,\dots,y_n)$ are $K$-linearly
  independent, by expressing them as $K$-linear combinations of
  $(1,z_1,\dots,z_n)$ (which are $K$-linearly independent), and
  proving that the coordinate matrix is invertible.

  For $i,j$ in $\lbrace 1, \ldots , n \rbrace$, let us define $G_{i,j} =
  \lbrace g \in G : g(z_i) = z_j \,\, \text{or} \,\, g(z_i) = 1-z_j
  \rbrace $. By the transitivity assumption, $G_{i,j}$ is
  non-empty for every $1 \leq i,j \leq n$. Let us further define $G^{+}_{i,j}
  = \lbrace g \in G : g(z_i) = z_j \rbrace$ and $G^{-}_{i,j} =
  \lbrace g \in G : g(z_i) = 1- z_j \rbrace$, so that $G_{i,j}=
  G^{+}_{i,j}\sqcup G^{-}_{i,j}$.  

  Let $M^*$ be the coordinate matrix of $(1, y_1, \ldots,
  y_n)$ with respect to the $K$-basis $(1, z_1, \ldots, z_n)$; 
  we  have to show that $\det (M^*) \neq 0$.
  By definition, for $i$ in $\{1,\dots,n\}$, we have
  \begin{align*}
y_i = \sum_{g \in G} g ({\alpha}z_i)&= \sum_{j=1}^n \Big(\sum_{g\in G^{+}_{i,j}}g(\alpha)z_j +\sum_{g\in G^{-}_{i,j}}g(\alpha)(1-z_j)\Big)\\
&=\sum_{j=1}^n\sum_{g\in G^{-}_{i,j}}g(\alpha)+ \sum_{j=1}^n\Big(\sum_{g\in G^{+}_{i,j}}g(\alpha) -\sum_{g\in G^{-}_{i,j}}g(\alpha)\Big)z_j.
  \end{align*}
For $i,j \in \lbrace1, \ldots , n \rbrace$, define $m_{i,j} =
\sum_{g\in G^{+}_{i,j}}g(\alpha) -\sum_{g\in G^{-}_{i,j}}g(\alpha)$
and $c_i = \sum_{j=1}^n\sum_{g\in G^{-}_{i,j}}g(\alpha)$. The matrix
$M^*$ is then
$$M^* = \begin{bmatrix}
1 & 0 & \cdots & 0\\
c_1 & m_{1,1} & \cdots & m_{1,n}\\
\vdots & \vdots &  & \vdots\\
c_{n} & m_{n,1} & \cdots	& m_{n,n}
\end{bmatrix}.
$$
Let us write
$$M = \begin{bmatrix}
 m_{1,1} & \cdots & m_{1,n}\\
 \vdots &  & \vdots\\
 m_{n,1} & \cdots	& m_{n,n}
\end{bmatrix}.$$
Since $\det(M^*) = \det (M)$, it is enough to show that the
determinant of $M$ is non-zero; this will be done using
Lemma~\ref{signdet}. We now check that the hypotheses of the lemma are
satisfied.

As before, let $\rho: G \longrightarrow \mathrm{S}_n$, $\rho(g) =
\rho_g$ be the group homomorphism that corresponds to the action of
$G$ on $\lbrace z_1, \ldots , z_n \rbrace$, so that $\rho_g(i) = j$ if
and only if $g$ is in $G_{i,j}$. We will show that the columns
$M_1,\dots,M_n$ of $M$ are permuted up sign by the action of $G$.

Let $h$ be in $G$
and  $i,j$ be in $\lbrace 1, \ldots, n \rbrace$. We 
can then write
$$h(m_{i,j}) = h\Big( \sum_{g\in G^{+}_{i,j}}g(\alpha) -\sum_{g\in
  G^{-}_{i,j}}g(\alpha)\Big) = \sum_{g\in G^{+}_{i,j}}hg(\alpha)
-\sum_{g\in G^{-}_{i,j}}hg(\alpha).$$ 
As in the proof of Theorem~\ref{theo:1}, we have
$hG_{i,j} = G_{i,\rho_h(j)}$, but more precisely, we can write
\begin{align}
\left \{
\begin{array}{ll}
  G^{+}_{i,\rho_{h}(j)}&= hG^{+}_{i,j}\\
G^{-}_{i,\rho_h(j)}&= hG^{-}_{i,j}
\end{array}\right .
\text{~if~} h \in G^+_{j,\rho_h(j)}
\quad\text{and}\quad
\left \{
\begin{array}{cl}
  G^{+}_{i,\rho_{h}(j)}&= hG^{-}_{i,j}\\
G^{-}_{i,\rho_h(j)}&= hG^{+}_{i,j}
\end{array}\right .
\text{~if~} h \in G^-_{j,\rho_h(j)}.
\end{align}
In the first case, we deduce
$$m_{i,\rho_h(j)} =  \sum_{g\in G^{+}_{i,\rho_h(j)}}g(\alpha) -\sum_{g\in G^{-}_{i,\rho_h(j)}}g(\alpha) 
                  =  \sum_{g\in G^{+}_{i,j}}hg(\alpha) -\sum_{g\in G^{-}_{i,j}}hg(\alpha)
=h(m_{i,j});$$
in the second case, we get
$$m_{i\rho_h(j)} = \sum_{g\in G^{+}_{i,\rho_h(j)}}g(\alpha)
-\sum_{g\in G^{-}_{i,\rho_h(j)}}g(\alpha) = \sum_{g\in
  G^{-}_{i,j}}hg(\alpha) -\sum_{g\in
  G^{+}_{i,j}}hg(\alpha)=-h(m_{i,j}).$$ In other words, $h(M_j) = \pm
M_{\rho_h(j)}$, so $G$ acts on the columns of $M$, and this action is
transitive by assumption.

Secondly, the first column $M_1$ has entries 
$$\sum_{g \in G^{+}_{i,1}}g(\alpha)- \sum_{g \in
  G^{-}_{i,1}}g(\alpha), i = 1, \ldots,n.$$ Since $\alpha$ is a normal
element of the Galois extension $K/F$ with Galois group $G$, and since
$G = \sqcup^n_{i =1}G_{i,j}= \sqcup^n_{i =1}(G^{+}_{i,1} \sqcup
G^{-}_{i,1}) $ is a disjoint union, this set is $F$-linearly
independent.

So Lemma \ref{signdet} applies, and conclude that $1, y_1, \ldots,
y_n$ are $K$-linearly independent. As mentioned above, this implies
that $K(x_1,\dots,x_n)^G=F(y_1,\dots,y_n)$.  This finishes the proof
in the transitive case; the proof in the general case follows in the
same manner as in Theorem~\ref{theo:1}.
\end{proof}


\begin{example}
Assume $G\leq \mathrm{GL}(3,\Z)$ generated by $$
\sigma = \begin{bmatrix}
0&-1&0\\
1&0&0\\
0&0&-1
\end{bmatrix}
,$$
so that $G \cong C_4$. Suppose $K= \Q(\rho)$, where $\rho$ is a primitive $5$-th root of unity, $K/\Q$ is Galois, with $\mathrm{Gal}(K/\Q) \cong C_4$. Let $x_1, x_2, x_3$ be algebraically independent over $K$. We want to find $K(x_1,x_2,x_3)^G$. 
%The action of $G$ on the $x_i$s and  the $5$-th roots of unity are given by  
%
%\begin{table}[H]
%\centering
%\begin{tabular}{l|llll} 
% & id & $\sigma$ & $\sigma^2$ & $\sigma^3$\\
% \hline
% $x_1$  & $x_1$ & $x_2^{-1}$ & $x_1^{-1}$ & $x_2$ \\
%$x_2$ & $x_2$ & $x_1$ & $x_2^{-1}$ & $x_1^{-1}$ \\
%$x_3$ & $x_3$ & $x_3^{-1}$ & $x_3$ & $x_3^{-1}$ \\
%$\rho$ & $\rho$ & $\rho^2$ & $\rho^4$ &  $\rho^3$\\
%$\rho^2$ & $\rho^2$ & $\rho^4$ & $\rho^3$ &  $\rho$\\
%$\rho^3$ & $\rho^3$ & $\rho$ & $\rho^2$ &  $\rho^4$\\
%$\rho^4$ & $\rho^4$ & $\rho^3$ & $\rho$ &  $\rho^2$\\
%\end{tabular}
%\end{table}
%\medskip

Define $$z_i = (1+x_i)^{-1}  \,\,\, \text{for} \, 1\leq i \leq 3. $$ 
%the action of $G$ on the $z_i$s is given by
%\begin{table}[H]
%\centering
%\begin{tabular}{l|llll} 
% & id & $\sigma$ & $\sigma^2$ & $\sigma^3$\\
% \hline
% $z_1$  & $z_1$ & $1-z_2$ & $1-z_1$ & $z_2$ \\
%$z_2$ & $z_2$ & $z_1$ & $1-z_2$ & $1-z_1$ \\
%$z_3$ & $z_3$ & $1-z_3$ & $z_3$ & $1-z_3$ \\
%\end{tabular}
%\end{table}
%\medskip

To form $y_i$ we need $S = 1+\sigma + \sigma^2 +\sigma ^3$. Then 
$$y_1 = S(\rho z_1) = \rho z_1 + \rho^2 (1-z_2) + \rho^4 (1-z_1) + \rho^3 z_2 = \rho^2+\rho^4 +(\rho - \rho^4)z_1 + (\rho^3 -\rho^2)z_2 $$ 
$$y_2 = S(\rho z_2) = \rho z_2 + \rho^2 z_1 + \rho^4 (1-z_2) + \rho^3 (1-z_1)= (\rho^3 + \rho^4) + (\rho^2 -\rho^3)z_1+ (\rho-\rho^4)z_2$$
$$y_3 = S(\rho z_3) = \rho z_3 + \rho^2 (1-z_3) + \rho^4 z_3 + \rho^3 (1-z_3)= (\rho^2+\rho^3) + (\rho-\rho^2-\rho^3+\rho^4)z_3.$$
Just to compare with the proof of the Theorem \ref{nonamesign}, the matrix $M^*$ is given 
$$
M^* = \begin{bmatrix}
1 & 0&0&0\\
\rho^2+\rho^4 & \rho -\rho^4 & \rho^3 -\rho^2 & 0\\
\rho^3+\rho^4 & \rho^2 -\rho^3 & \rho -\rho^4 &0 \\
\rho^2+\rho^3 &0 &0 & \rho - \rho^2 -\rho^3+\rho^4
\end{bmatrix}
$$
 and 
 $$
 M = \begin{bmatrix}
 \rho -\rho^4 & \rho^3 -\rho^2 & 0\\
 \rho^2 -\rho^3 & \rho -\rho^4 &0 \\
0 &0 & \rho - \rho^2 -\rho^3+\rho^4
\end{bmatrix}.
$$
One can verify the action of $G$ on the columns of $M$ is 
\begin{table}[H]
\centering
\begin{tabular}{l|llll} 
 & id & $\sigma$ & $\sigma^2$ & $\sigma^3$\\
 \hline
 $M_1$  & $M_1$ & $-M_2$ & $-M_1$ & $M_2$ \\
$M_2$ & $M_2$ & $M_1$ & $-M_2$ & $-M_1$ \\
$M_3$ & $M_3$ & $-M_3$ & $M_3$ & $-M_3$ \\
\end{tabular}
\end{table}
\end{example}


\bibliographystyle{plain}
\bibliography{bibliography}

\end{document}
