\documentclass[12pt]{article}

%%%%%%%%%%%%%%%% 
\usepackage{fullpage}
\usepackage{appendix}
\usepackage{graphicx}
\usepackage{amsmath}
\usepackage{amsthm}
\usepackage[byname]{smartref}
\usepackage{hyperref} 
\usepackage{tocloft}
\usepackage{tikz}
\usepackage{xcolor}
\usepackage{listings}
%\usepackage{mathtools}
\usetikzlibrary{positioning}
\usepackage{tikz-qtree,tikz-qtree-compat}
\usetikzlibrary{trees}
\usetikzlibrary{arrows,positioning,automata,shadows,fit,shapes}
\usepackage{mathtools}
\usepackage{float}
\usepackage{amssymb}
\usepackage{tikz-cd}
%\usepackage{pagecolor,lipsum}
%\pagecolor{lightgray}
\usepackage[english]{babel}
%\usepackage[table]{xcolor}
\usepackage{algorithmic}
\usepackage{algorithm}
\usepackage{authblk}
\usepackage{tikz-cd}
%\usepackage{chngcntr}
\renewcommand{\algorithmicrequire}{\textbf{Input:}}
\renewcommand{\algorithmicensure}{\textbf{Output:}}
%%%%%%%%%%%%%%%%%%%%%%%%%%%%
\usepackage{graphicx}
\usepackage{amsmath}
\usepackage{amsfonts}
\usepackage{mathrsfs}
\usepackage{bm}


%\hypersetup{
%	unicode = true,
%	colorlinks = true,
%	citecolor = blue,
%	filecolor = black,
%	linkcolor = black,
%	urlcolor = black,
%	pdfstartview = {FitH},
%}



\theoremstyle{plain}
\newtheorem{theorem}{Theorem}
\newtheorem{lemma}[theorem]{Lemma}
\newtheorem{corollary}[theorem]{Corollary}
\newtheorem{proposition}[theorem]{Proposition}
\newtheorem{definition}[theorem]{Definition}
\newtheorem{conjecture}[theorem]{Conjecture}
\newtheorem{example}[theorem]{Example}
\newtheorem*{remark}{Remark}
\newtheorem{note}[theorem]{Note}


\def\todo#1{(\textbf{todo:} #1)}
\newcommand{\N}{\ensuremath{\mathbb{N}}}
\newcommand{\F}{\ensuremath{\mathbb{F}}}
\newcommand{\Z}{\ensuremath{\mathbb{Z}}}
\newcommand{\Q}{\ensuremath{\mathbb{Q}}}
\newcommand{\C}{\ensuremath{\mathcal{C}}}
\newcommand{\Po}{\ensuremath{\mathcal{P}}}
\newcommand{\G}{G}
\newcommand{\glat}{$G$-lattice}
\newcommand{\tand}{\ensuremath{\,\,\, \text{and} \,\,\,}}
\newcommand{\exactseq}[1]{\ensuremath{0 \longrightarrow M_{#1} \longrightarrow L_{#1} \longrightarrow \Z \longrightarrow 0}}
\newcommand{\exactseqs}[1]{\ensuremath{0 \longrightarrow M_{#1} \longrightarrow L_{#1} \longrightarrow \Z^{-} \longrightarrow 0}}
\newcommand*{\QEDA}{\hfill\ensuremath{\blacksquare}}
\DeclareBoldMathCommand{\be}{e}
\DeclareBoldMathCommand{\bx}{x}
\DeclareBoldMathCommand{\by}{y}


%%%%%%%%%%%%%%%%

\title{Algebraic Construction of Quasi Split Algebraic Tori}
\author{}

\begin{document}

\maketitle


\begin{abstract}

The main purpose of this work is to give a constructive proof for the
No Name Lemma. \todo{or of a particular case of it?} Let $G$ be a
finite group, $K$ be a field, $L$ be a permutation $G$-lattice with
the standard basis and $K[L]$ be the group algebra of $L$ over
$K$. The No Name Lemma asserts that the invariant field of the
quotient field of $K[L]$, $K(L)^G$ is a purely transcendental
extension of $K^G$. In other words, there exist $y_1, \ldots , y_n$
which are algebraically independent over $K^G$ such that $K(L)^G \cong
K^G(y_1, \ldots , y_n)$. For a Galois extension $K/F$ with $G =
\mathrm{Gal}(K/F)$ we introduce $\mathcal{Y}=\lbrace y_1, \ldots, y_n
\rbrace \subset K[L]^G$ with desired properties. For a sign
permutation $G$-lattice $L$, a more general argument is given so that
we can concretely find a transcendence basis of $K(L)^G$ over
$K^G$.  \todo{Improve}
\end{abstract}

\section{Introduction}



An algebraic $F$-torus $T$ is an algebraic group defined over a field
$F$ which {\em splits} over an algebraic closure $\bar F$ of $F$, which is
 isomorphic to a torus (a finite product of copies of the
multiplicative group $\mathbb{G}_m$) over $\bar{F}$. In general,
$\bar{F}$ is not the smallest field over which $T$ splits: it is known
that an algebraic $F$-torus $T$ splits over a finite Galois extension
of $F$. There is a unique minimal such extension, say $K$; if $G =
\mathrm{Gal}(K/F)$, then $G$ is called the splitting group of $T$. For
more details, see~\cite[Chapter 2]{Voskresenskii}.


A $G$-lattice $L$ is a free $\Z$-module $A$ of finite rank, together
with a group homomorphism $G \longrightarrow {\rm Aut}(A)$ (the group
of automorphisms of $A$). Given a module basis of $A$, any group
homomorphism $G \longrightarrow {\rm GL}(n,\Z)$, with $n={\rm
  rank}(A)$, gives such an action.  If $K$ is a field, the group
algebra $K[L]$ of $L$ over $K$ is isomorphic to the $K$-algebra of
Laurent polynomials $K[x_1^{\pm 1},\dots,x_n^{\pm 1}]$, for some
indeterminates $x_1,\dots,x_n$.  If $K$ is equipped with an action of
$G$ (that is, a $G$-field), we can extend the action of $G$ on lattice
$L$ to an action on $K[L]$; the ring $K[L]^G$ of {\em multiplicative
  invariants} consists of those elements in $K[L]$ invariant under the
action of $G$. The fraction field $K(L)$ of $K[L]$ is isomorphic to
$K(x_1,\dots,x_n)$, and the subfield of invariants under the action of
$G$ is written $K(L)^G$.

It is known that there is a duality between the category of algebraic
tori with splitting group $G$ and $G$-lattices. For a given algebraic
torus $T$ with splitting group $G$, its character module
$\mathrm{Hom}(T,\mathbb{G}_m)$ is a $G$-lattice. Conversely, if $L$ is
a $G$-lattice, with $G={\rm Gal}(K/F)$ for some finite Galois
extension $K/F$, then $T=\mathrm{Spec}(K[L]^G)$ is an algebraic
$F$-torus with splitting group $G$, with coordinate ring $K[L]^G$ and
function field $K(L)^G$.

A $G$-lattice $L$ is called {\em permutation} (resp.\ {\em sign
  permutation}) if it has a $\Z$-basis which is permuted (resp.\ up to
sign changes) by $G$. The No Name Lemma asserts that if $L$ is a
permutation $G$-lattice and $K$ is a $G$-field, then $K(L)^G$ is
rational over $K^G$~\cite[Chapter~9.4]{Lorenz}; in particular, with
$G={\rm Gal}(K/F)$, $K(L)^G$ is $F$-rational.

In this paper, we give a constructive proof of this fact, by
exhibiting a basis for such a field of invariants.  In concrete terms,
we start from a subgroup of $\mathrm{GL}(n,\Z)$ and describe the field
of functions of an associated torus.

\begin{definition}\label{Assumption}
  Let $G$ be a finite subgroup of $\mathrm{GL}(n,\Z)$.  The
  $G$-lattice $L_G$ corresponding to $G$ is the rank $n$ lattice
  generated by the standard basis $\langle \be_i : i = 1,
  \ldots, n \rangle_\Z$, where $\be_i$ is the column vector
  $[\delta_{i,j}, i=1,\dots,n]^T$, together with the action of $G$ given
  by left-multiplication on the $\be_i$'s.
\end{definition}

Suppose further that we are given an isomorphism $\iota: G \to
\mathrm{Gal}(K/F)$, for some finite Galois extension $K/F$ (in what
follows, we simply say that $K/F$ has Galois group $G$). Then, through
the identification $K[x_1^{\pm 1}, \ldots , x_n^{\pm 1}]\simeq K[L]$,
$K[x_1^{\pm 1}, \ldots , x_n^{\pm 1}]$ is equipped with the $G$-action
as follows:
\begin{itemize}
\item for $g$ in $G$ and $\alpha$ in $K$, we write
  $g(\alpha)=(\iota(g))(\alpha)$ (so $G$ acts as the Galois group on
  $K$);
\item for $g$ in $G$ and $i=1,\dots,n$, $g(x_i) = \prod_{j=1}^{j=n}
  x_j^{g_{i,j}}$, where $g_{i,j}$ is the $(i,j)$th -entry of $g$
 (so that we also have $g(\be_i) =
  \sum_{j=1}^{n} g_{i,j}\be_j$).
\end{itemize}
If we let $T_G$ be the algebraic torus corresponding to $L_G$, then
$T_G$ is an algebraic $F$-torus which splits over $K$, with character
lattice $L_G$ and function field $K(x_1,\dots,x_n)^G$.

Conjugate subgroups of $\mathrm{GL}(n,\Z)$ correspond to isomorphic
lattices, and isomorphic algebraic tori; in particular, for $G$ a
finite subgroup of $\mathrm{GL}(n,\mathbb{Z})$, $L_G$ is a (signed)
permutation lattice if and only if $G$ is conjugate to a group of
(signed) permutation matrices.  Computationally, we do not have an
efficient algorithm at hand to decide whether a given lattice is
(signed) permutation. Hence, in our main results, we will assume that
$G$ is a subgroup of the group $\mathbb{S}_n$ of permutation matrices
of size $n$, or more generally of the group $\mathbb{B}_n$ of signed
permutation matrices of size $n$.  In such a case, for $i$ in
$\{1,\dots,n\}$ and $g$ in $G$, $g(\be_i)=\pm \be_j$ for some index
$j$ in $\{1,\dots,n\}$ (all signs being $+1$ if $G$ is a subgroup of
$\mathbb{S}_n$), and the action of $g \in G$ on $x_i$ is given
by $$g(x_i) = \begin{cases} x_j & \text{if~} g(\be_i) = \be_j
  \\ x^{-1}_j & \text{if~} g(\be_i) = -\be_j. \end{cases}$$

For such groups $G$, the $F$-rationality of the torus $T_{G}$ means
that for $K(x_1,\dots,x_n)$, endowed with the $G$-action we just
described, there exist $y_1,\dots,y_n$ such that
$K(x_1,\dots,x_n)^G=F(y_1,\dots,y_n)$. However, the proofs of the
No Name Lemma we are aware of~\cite{xxx} are non constructive.  The
goal of this paper is to exhibit such $y_i$'s; we state two such
results. \todo{Better discussion needed. How general are our results
compared to other version of the no-name lemma?}

In both our theorems, we rely on the notion of {\em normal element} of
a finite Galois extension $K/F$ with Galois group $G$; we recall that
$\alpha \in K$ is normal if $\alpha$ and all its Galois conjugates
form an $F$-basis of $K$.  Any finite Galois extension admits a normal
element~\cite[Theorem 6.13.1]{Lang}; there exist algorithms to
construct one, in characteristic zero~\cite{Girstmair} or in positive
characteristic~\cite{Giesbrecht,Poli}.

\begin{theorem}\label{nonamenonsign}
  Let $G$ be a subgroup of $\mathbb{S}_n$, let $K/F$ be a finite
  Galois extension with Galois group $G$, and let $\alpha \in K$ be a
  normal element for $K/F$. Then $ K(x_1,\ldots ,
  x_n)^G=F(y_1,\dots,y_n)$, with
  $$y_i =\sum_{g \in G} g(\alpha x_i), \quad i = 1,\dots,n.$$
\end{theorem}

Our second statement is similar, but deals with the more general case
of signed permutations (if the group $G$ below happens to be non-signed
permutation, the construction is not the same as in the previous theorem).

\begin{theorem}\label{nonamesign}
  Let $G$ be a subgroup of $\mathbb{B}_n$, let $K/F$ be a finite
  Galois extension with Galois group $G$, and let $\alpha \in K$ be a
  normal element for $K/F$. Then
  $K(x_1,\ldots, x_n)^G = F(y_1, \ldots, y_{n}),$
  with $$y_i = \sum_{g \in G} g\left (\frac{\alpha}{1+x_i}\right), \quad i = 1,\dots,n.$$
\end{theorem}

There are many algorithms for finding the invariant rings for
polynomial invariants~\cite{Sturmfels,Kemper2}. For multiplicative
invariants, the algorithmic landscape is not developed to the same
extent; see however some references in~\cite{Kemper,Lorenz,Renault}.
\todo{say more}

%%%%%%%%%%%%%%%%%%%%%%%%%%%%%%%%%%%%%%%%%%%%%%%%%%%%%%%%%%%%
%%%%%%%%%%%%%%%%%%%%%%%%%%%%%%%%%%%%%%%%%%%%%%%%%%%%%%%%%%%%
%%%%%%%%%%%%%%%%%%%%%%%%%%%%%%%%%%%%%%%%%%%%%%%%%%%%%%%%%%%%

\section{Proofs and examples}


In what follows, we use the following notation: we still write
$\mathbb{S}_n$, resp.\ $\mathbb{B}_n$, for the groups of permutation,
resp.\ signed permutation matrices of size $n$; we let
$\mathfrak{S}_n$ be the symmetric group of size $n$ and we denote by
$\rho: \mathbb{B}_n \to \mathfrak{S}_n$ the group homomorphism
obtained by mapping a signed permutation matrix to the permutation
$\rho_g$ such that $\rho_g(i)=j$, where $j$ is the index of the
unique non-zero entry in the $i$th column of $g$.  Hence, in terms of
the action defined in the previous section, for $g$ in $\mathbb{B}_n$
and for all $i,j$ in $\{1,\dots,n\}$, we have $g(x_i) =
x_{\rho_g(i)}^{\pm 1}$.  For $i,j$ as above, we also denote by
$G_{i,j}$ the set of all $g$ in $G$ such that $g(x_i)=x^{\pm 1}_j$, that is,
such that $\rho_g(i)=j$.

We start with a lemma that generalizes known facts about Moore
matrices over finite fields (see Example~\ref{ex:Moore} below). Let
$G$ be a subgroup of $\mathbb{B}_n$ and let $K/F$ be a finite Galois
extension with Galois group isomorphic to $G$, as in the previous
section. Through this isomorphism, $G$ acts on (column) vectors
entrywise: for $g \in G$ and a column vector in $K^m$, written as $C =
\left[\mu_{1} \,\,\, \mu_{2} \,\,\, \cdots \,\,\, \mu_{m}\right]^T$,
we define $g(C) = \left[g(\mu_{1}) \,\,\, g(\mu_{2}) \,\,\, \cdots
  \,\,\, g(\mu_{m})\right]^T$.

Let then $M$ be in $M_{n,n}(K)$. For such a matrix, and for
$j=1,\dots,n$, its $j$th column is written $M_j=\left[\mu_{1,j} \,\,\,
  \mu_{2,j} \,\,\, \cdots \,\,\, \mu_{m,j}\right]^T$.  We say that $G$
{\em permutes the columns of $M$ up to sign} if for $g$ in $G$ and $j$
in $\{1,\dots,n\}$, we have $g(M_j)= \pm M_{\rho_g(j)}$.
 
\begin{lemma}\label{signdet}
   Let $K/F$ be a finite Galois extension with Galois group $G$. Let
   $M$ be in $M_{n,n}(K)$ and assume that $G$ permutes the columns of
   $M$ up to sign. Assume also that the entries of the first column
 of $M$ are $F$-linearly independent. Then $M$ is invertible.
\end{lemma}
\begin{proof}
  Assume by contradiction that there is a non-zero vector in the left
  nullspace of $M$; take $\bx$ be a vector with the minimum number of
  non-zero entries among the non-zero left nullspace elements.
  Let $k \in \{1,\dots,m\}$ be such that $x_k \neq 0$ and let
  $\by=\frac 1{x_k} \bx \in K^n$, so that $y_k = 1$, and $\by$ is
  still in the left nullspace of $M$. 
  
  For $i$ in $\{1,\dots,n\}$, we have the equality $\by M_i = 0 $,
  where $M_i$ is the $i$th column of $M$. For $g$ in $G$, we deduce
  $g(\by M_i) = g(\by)g(M_i) = \pm g(\by) M_{\rho_g(i)} = 0$. Since
  this is true for all $i$, we obtain that $g(\by)$ is in the
  left-nullspace of $M$ as well. This further implies that
  $\by'=g(\by)-\by$ is in the nullspace of $M$. However, since
  $y_k=1$, $g(y_k)=1$, so that $y'_k=0$. By construction of $\bx$, this
  implies that $\by'=0$, so that $g(\by)=\by$. 

  Since this is true for all $g$, we deduce that $\by$ is in $F^n$.
  Then, the relation $\by M_1=0$ implies that $\by=0$, a
  contradiction.
 \end{proof} 


\begin{example}\label{ex:Moore}
  Let $F=\F_q$, for some prime power $q$, and let $K=\F_{q^n}$.  The
  Galois group of $K/F$ is cyclic of size $n$, generated by the Frobenius
  map $x \mapsto x^q$. Let then $(\alpha_1,\dots,\alpha_n)$ be in $K$,
  and consider the matrix $M=[m_{i,j}]_{1 \le i,j \le n}$, with
  $m_{i,j} = \alpha_i^{q^{j-1}}$. The Frobenius map permutes the
  columns of $M$, and we recover the fact that if
  $(\alpha_1,\dots,\alpha_n)$ are $F$-linearly independent, $M$ is
  invertible~\cite[Corollary~1.3.4]{Goss}
\end{example}


We can now prove our first result.

%% \begin{theorem}\label{theo:1}
%%   Let $G$ be a subgroup of $\mathbb{S}_n$, let $K/F$ be a finite
%%   Galois extension with Galois group $G$, and let $\alpha \in K$ be a
%%   normal element for $K/F$. Then $ K(x_1,\ldots ,
%%   x_n)^G=F(y_1,\dots,y_n)$, with
%%   $$y_i =\sum_{g \in G} g(\alpha x_i), \quad i =
%%   1,\dots,n.$$
%% \end{theorem}
\begin{proof}[Proof of Theorem~\ref{nonamenonsign}]
  We first prove the result under the additional assumption that $G$ acts
  transitively on $L_G$.  The elements $(y_1,\dots,y_n)$, with $y_i =
  \sum_{g \in G} g(\alpha x_i)$ as defined in the theorem, are invariant under
  the action of $G$. We will show below that they are $K$-linearly
  independent; this will prove that
  $K(x_1,\dots,x_n)^G=F(y_1,\dots,y_n)$, since then
  $K(x_1,\dots,x_n)=K(y_1,\dots,y_n)$, and
  $K(y_1,\dots,y_n)^G=K^G(y_1,\dots,y_n)=F(y_1,\dots,y_n)$.

  For $i,j$ in $\{1,\dots,n\}$, let $G_{i,j}= \lbrace g \in G: g(x_i)
  = x_j \rbrace$, so that we can rewrite $y_i$ as $$y_i = \sum^n_{j
    =1}\sum_{g \in G_{i,j}}g(\alpha)x_j, \,\,\, i = 1, \ldots, n.$$
  Since the action of $G$ on $L_G$ is transitive, $G_{i,j}$ is
  non-empty for every $1 \leq i,j \leq n$.  Take such indices $i,j$,
  and fix some $g_{i,j}$ in $G_{i,j}$. If $g \in G_{i,j}$, then
  $g^{-1}_{i,j}g(x_i) = x_i$ shows that $g$ is in
  $g_{i,j}\mathrm{Stab}_G(x_i)$. Since we also have
  $g_{i,j}\mathrm{Stab}_G(x_i) \subseteq G_{i,j}$, we see that
  $G_{i,j} = g_{i,j}\mathrm{Stab}_G(x_i)$.


  We now show that the matrix $M$ with $i$th row the coordinate vector
  of $y_i$ with respect to the $K$-basis $\lbrace x_1, \ldots x_n
  \rbrace$ is invertible. The matrix $M$ has entries $m_{i,j} =
  \sum_{g \in G_{i,j}}g(\alpha)$, $i,j = 1, \ldots, n$. We will apply
  Lemma~\ref{signdet} to show that $M$ has $K$-linearly independent
  rows.

  We check the hypothesis of the lemma. First, let $\rho: G \to
  \mathfrak{S}_n$, $\rho(g) = \rho_g$ be the group homomorphism that
  corresponds to the action of $G$ on $(x_1, \ldots , x_n)$, so that
  $\rho_g(i) = j$ if and only if $g(x_i) = x_j$ for all $1 \leq
  i,j\leq n$. We will show that the columns of $M$ are permuted by the
  action of $G$. Let thus $h$ be in $G$. Note that for $g$ in
  $G_{i,j}$, $hg$ is in $G_{i,\rho_h(j)}$; since $G_{i,\rho_h(j)}$ and
  $G_{i,j}$ have the same cardinality, equal to
  $|\mathrm{Stab}_G(x_i)|$, we get
  $$h(m_{i,j}) = \sum_{g \in G_{i,j}}hg(\alpha) = \sum_{\sigma \in
    G_{i, \rho_{h}(j)}}\sigma (\alpha) = m_{i,\rho_{h}}(j).$$ This
  shows that $h(M_j) = M_{\rho_h(j)}$ for all $j = 1, \ldots, n$, so
  that $G$ permutes the columns of $M$. 
  
  Finally, the first column $M_1$ has entries $[ \sum_{g \in
      G_{i,1}}g(\alpha),\ i = 1, \ldots,n ].$ Since $\alpha$ is a
  normal element of the Galois extension $K/F$ with Galois group $G$,
  the set $\lbrace g(\alpha): g \in G \rbrace$ is $F$-linearly
  independent. Since $G = \sqcup^n_{i =1}G_{i,1}$ is a disjoint union,
  and all $G_{i,1}$ are non-empty,
  the set
  $$\left\lbrace \sum_{g \in G_{i,1}}g(\alpha), i = 1, \ldots, n \right\rbrace$$ is
  $F$-linearly independent as well.
  So Lemma~\ref{signdet} applies, and we conclude that $y_1, \ldots,
  y_n$ are $K$-linearly independent, as claimed.

  \medskip 
  
  We can now give the proof of our claim in the general case.  Let
  $\be_1, \ldots , \be_n$ be the standard basis of $L_G$, and let
  $\{\be_{j_k} \mid k = 1, \ldots, r\}$ and correspondingly $\{x_{j_k}
  \mid k = 1, \ldots, r\}$ be a complete set of $G$-orbit
  representatives among the basis vectors, and the indeterminates
  $x_1, \ldots, x_n$ respectively. Then $L_k = \oplus_{ \be_i \in G
    \be_{j_{k}} } \Z \be_i$ is a transitive permutation $G$-lattice
  for each $k = 1, \ldots, r$, and $K(L_k) = K(x_i \mid x_i \in
  Gx_{j_k})$.
  
  The lattice $L_G = \oplus^r_{k =1} P_k$ is a direct sum of transitive
  permutation $G$-lattices, and similarly $K(x_1,\dots,x_n)^G$ is the
  compositum of the fields $K(P_k)$, $k=1,\dots,r$. Thus, using the 
  result established in the transitive case, we obtain
  $K(x_1,\dots,x_n)^G=F(y_i \mid x_i \in Gx_{j_k}, k=1,\dots,r)$, where
  for all $k$ and for $x_i \in Gx_{j_k}$, we have $y_i=\sum_{g \in G}
  g(\alpha x_i)$.
\end{proof}



\begin{example}
Let $K$ be the splitting field of $x^4-2$ over $F=\Q$. Then
$\mathrm{Gal}(K/\Q) \cong D_8$, $K = \Q(\theta,i)$, with $\theta =
\sqrt[4]{2}$, and $\lbrace 1, \theta, \theta^2, \theta^3, i, i\theta,
i\theta^2, i\theta^3\rbrace$, is a $\Q$-basis for $K$. Let
$n=4$, let
$G\leq \mathrm{GL}(4,\Z)$ be generated by
$$
r = \begin{bmatrix}
0&0&0&1\\
1&0&0&0\\
0&1&0&0\\
0&0&1&0
\end{bmatrix} \,\,\,\, \text{and}\,\,\,\,
s = \begin{bmatrix}
1&0&0&0\\
0&0&0&1\\
0&0&1&0\\
0&1&0&0
\end{bmatrix}.
$$ and let $(x_1,\dots,x_4)$ be new indeterminates, on which $G$ acts as in
Definition~\ref{Assumption}; this action is transitive. One can verify
that $G$ is isomorphic to $\mathrm{Gal}(K/\Q)$; through this isomorphism,
the action of $r$ and $s$ on the generators of $K$ is given by
$$
\begin{matrix}
r(i) = i &r(\theta) = i \theta \\
s(i) = -i & s(\theta) = \theta.
\end{matrix}
$$
Now, define $$\alpha = 1+ \theta + \theta^2 + \theta^3 +i + i\theta + i\theta^2 + i\theta^3;$$
this is a normal element in $K$. 
The elements $(y_1,\dots,y_4)$ of Theorem~\ref{nonamenonsign}, expressed 
on the basis $(x_1,\dots,x_4)$, are given by the coordinate matrix
$$
M= \begin{bmatrix}
(1+s)(\alpha) & (r^3+sr)(\alpha) & (r^2+sr^2)(\alpha) & (r+sr^3)(\alpha)\\
(r+sr)(\alpha) & (1+sr^2)(\alpha) & (r^3+sr^3)(\alpha) & (r^2+s)(\alpha)\\
(r^2+sr^2)(\alpha) & (r+sr^3)(\alpha) & (1+s)(\alpha) & (r^3+sr)(\alpha)\\
(r^3+sr^3)(\alpha) & (r^2+s)(\alpha) & (r+sr)(\alpha) & (1+sr^2)(\alpha)
\end{bmatrix}.
$$
The action of $r$ and $s$ on the columns is given by
%% $$r(M_1)=M_4,\quad r(M_2)=M_1,\quad r(M_3)=M_2, \quad r(M_4)=M_3$$
%% and
%% $$s(M_1)=M_1,\quad s(M_2)=M_4,\quad s(M_3)=M_3, \quad s(M_4)=M_2.$$
\begin{table}[H]
\centering
\begin{tabular}{l|llllllll} 
 & $r$ & $s$ \\
 \hline
 $M_1$  & $M_4$ & $M_1$ \\
$M_2$ & $M_1$ & $M_4$ \\
$M_3$ & $M_2$ & $M_3$ \\
$M_4$ & $M_3$ & $M_2$ 
\end{tabular}
\end{table}
\end{example} 

\begin{remark}
With the assumptions of the previous theorem, we can actually compute
the coordinate ring of the torus; we obtain $$K[L_G]^G \cong K[x^{\pm
    1}_1, x^{\pm 1}_2, \ldots , x^{\pm 1}_n]^G = F[y_1, \ldots ,
  y_n]_{x_1\cdots x_n},$$ for $y_1,\dots,y_n$ as in the theorem.
Indeed, we have $K[x^{\pm 1}_1, x^{\pm 1}_2, \ldots , x^{\pm 1}_n] =
K[x_1, \ldots , x_n]_{x_1\cdots x_n}.$ We are interested in
$K[L_G]^G$, that is, $\left( K[x^{\pm 1}_1, x^{\pm 1}_2, \ldots ,
  x^{\pm 1}_n] \right)^G = \left(K[x_1, \ldots , x_n]_{x_1\cdots x_n}
\right) ^G.$ The proof of Theorem~\ref{nonamenonsign} shows that
$K[x_1, \ldots , x_n] = K[y_1, \ldots , y_n]$.  On the other hand
since $G$ permutes the $x_i$'s, $x_1\cdots x_n$ is invariant under the
action of $G$, and we can conclude $$\left( K[x_1, \ldots ,
  x_n]_{x_1\cdots x_n}\right)^G = \left( K[y_1, \ldots ,
  y_n]_{x_1\cdots x_n} \right)^G = K^G [y_1, \ldots , y_n]_{x_1\cdots
  x_n} = F[y_1, \ldots , y_n]_{x_1\cdots x_n}.$$ One could further
rewrite $x_1\cdots x_n$ as a degree $n$ homogeneous polynomial in
$y_1,\dots,y_n$ (but the expression obtained this way is not
particularly handy).
\end{remark}

We conclude with the proof of our second main result.  The proof
follows that of Theorem~\ref{nonamenonsign}, the only difference being in
the description of the coordinate matrix $M$.  As in
Theorem~\ref{nonamenonsign}, we first prove the result under the extra
assumption that that $G$ acts transitively up to sign on $L_G$.

\begin{proof}[Proof of Theorem~\ref{nonamesign}]
  Assume first that the action of $G$ is transitive (up to sign).
  For $i$ in $\{1,\dots,n\}$, define $z_i = (1+x_i)^{-1} $. Now for $g
  \in G$, $$g(z_i) = \begin{cases} z_j & \text{if} \,\,\, g(x_i) = x_j
    \\ 1-z_j & \text{if} \,\,\, g(x_i) = x_j^{-1},
  \end{cases}$$ 
  and $K(x_1, \ldots , x_n) = K(z_1, \ldots, z_n).$ The elements $y_i$
  can be rewritten as $y_i = \sum_{g \in G} g ({\alpha}z_i)$, for $i$
  in $\{1, \ldots, n\}$; as before, in order to prove that
  $K(z_1,\ldots, z_n)^G = F(y_1, \ldots, y_{n})$, it is enough to
  prove that $y_1,\dots,y_n$ are $K$-linearly independent.

  We actually prove that $(1,y_1,\dots,y_n)$ are $K$-linearly
  independent, by expressing them as $K$-linear combinations of
  $(1,z_1,\dots,z_n)$ (which are $K$-linearly independent), and
  proving that the coordinate matrix is invertible.

  For $i,j$ in $\lbrace 1, \ldots , n \rbrace$, let us define $G_{i,j} =
  \lbrace g \in G : g(z_i) = z_j \,\, \text{or} \,\, g(z_i) = 1-z_j
  \rbrace $. By the transitivity assumption, $G_{i,j}$ is
  non-empty for every $1 \leq i,j \leq n$. Let us further define $G^{+}_{i,j}
  = \lbrace g \in G : g(z_i) = z_j \rbrace$ and $G^{-}_{i,j} =
  \lbrace g \in G : g(z_i) = 1- z_j \rbrace$, so that $G_{i,j}=
  G^{+}_{i,j}\sqcup G^{-}_{i,j}$.  

  Let $M^*$ be the coordinate matrix of $(1, y_1, \ldots,
  y_n)$ with respect to the $K$-basis $(1, z_1, \ldots, z_n)$; 
  we  have to show that $\det (M^*) \neq 0$.
  By definition, for $i$ in $\{1,\dots,n\}$, we have
  \begin{align*}
y_i = \sum_{g \in G} g ({\alpha}z_i)&= \sum_{j=1}^n \Big(\sum_{g\in G^{+}_{i,j}}g(\alpha)z_j +\sum_{g\in G^{-}_{i,j}}g(\alpha)(1-z_j)\Big)\\
&=\sum_{j=1}^n\sum_{g\in G^{-}_{i,j}}g(\alpha)+ \sum_{j=1}^n\Big(\sum_{g\in G^{+}_{i,j}}g(\alpha) -\sum_{g\in G^{-}_{i,j}}g(\alpha)\Big)z_j.
  \end{align*}
For $i,j \in \lbrace1, \ldots , n \rbrace$, define $m_{i,j} =
\sum_{g\in G^{+}_{i,j}}g(\alpha) -\sum_{g\in G^{-}_{i,j}}g(\alpha)$
and $c_i = \sum_{j=1}^n\sum_{g\in G^{-}_{i,j}}g(\alpha)$. The matrix
$M^*$ is then
$$M^* = \begin{bmatrix}
1 & 0 & \cdots & 0\\
c_1 & m_{1,1} & \cdots & m_{1,n}\\
\vdots & \vdots &  & \vdots\\
c_{n} & m_{n,1} & \cdots	& m_{n,n}
\end{bmatrix}.
$$
Let us write
$$M = \begin{bmatrix}
 m_{1,1} & \cdots & m_{1,n}\\
 \vdots &  & \vdots\\
 m_{n,1} & \cdots	& m_{n,n}
\end{bmatrix}.$$
Since $\det(M^*) = \det (M)$, it is enough to show that the
determinant of $M$ is non-zero; this will be done using
Lemma~\ref{signdet}. We now check that the hypotheses of the lemma are
satisfied.

As before, let $\rho: G \to \mathfrak{S}_n$, $\rho(g) =
\rho_g$ be the group homomorphism that corresponds to the action of
$G$ on $\lbrace z_1, \ldots , z_n \rbrace$, so that $\rho_g(i) = j$ if
and only if $g$ is in $G_{i,j}$. We will show that the columns
$M_1,\dots,M_n$ of $M$ are permuted up sign by the action of $G$.

Let $h$ be in $G$
and  $i,j$ be in $\lbrace 1, \ldots, n \rbrace$. We 
can then write
$$h(m_{i,j}) = h\Big( \sum_{g\in G^{+}_{i,j}}g(\alpha) -\sum_{g\in
  G^{-}_{i,j}}g(\alpha)\Big) = \sum_{g\in G^{+}_{i,j}}hg(\alpha)
-\sum_{g\in G^{-}_{i,j}}hg(\alpha).$$ 
As in the proof of Theorem~\ref{nonamenonsign}, we have
$hG_{i,j} = G_{i,\rho_h(j)}$, but more precisely, we can write
\begin{align}
\left \{
\begin{array}{ll}
  G^{+}_{i,\rho_{h}(j)}&= hG^{+}_{i,j}\\
G^{-}_{i,\rho_h(j)}&= hG^{-}_{i,j}
\end{array}\right .
\text{~if~} h \in G^+_{j,\rho_h(j)}
\quad\text{and}\quad
\left \{
\begin{array}{cl}
  G^{+}_{i,\rho_{h}(j)}&= hG^{-}_{i,j}\\
G^{-}_{i,\rho_h(j)}&= hG^{+}_{i,j}
\end{array}\right .
\text{~if~} h \in G^-_{j,\rho_h(j)}.
\end{align}
In the first case, we deduce
$$m_{i,\rho_h(j)} =  \sum_{g\in G^{+}_{i,\rho_h(j)}}g(\alpha) -\sum_{g\in G^{-}_{i,\rho_h(j)}}g(\alpha) 
                  =  \sum_{g\in G^{+}_{i,j}}hg(\alpha) -\sum_{g\in G^{-}_{i,j}}hg(\alpha)
=h(m_{i,j});$$
in the second case, we get
$$m_{i\rho_h(j)} = \sum_{g\in G^{+}_{i,\rho_h(j)}}g(\alpha)
-\sum_{g\in G^{-}_{i,\rho_h(j)}}g(\alpha) = \sum_{g\in
  G^{-}_{i,j}}hg(\alpha) -\sum_{g\in
  G^{+}_{i,j}}hg(\alpha)=-h(m_{i,j}).$$ In other words, $h(M_j) = \pm
M_{\rho_h(j)}$, so $G$ acts on the columns of $M$.
Secondly, the first column $M_1$ has entries 
$$\sum_{g \in G^{+}_{i,1}}g(\alpha)- \sum_{g \in
  G^{-}_{i,1}}g(\alpha), i = 1, \ldots,n.$$ Since $\alpha$ is a normal
element of the Galois extension $K/F$ with Galois group $G$, and since
$G = \sqcup^n_{i =1}G_{i,j}= \sqcup^n_{i =1}(G^{+}_{i,1} \sqcup
G^{-}_{i,1}) $ is a disjoint union, with all $G_{i,j}$ non-empty (by
the transitivity of the action), this set is $F$-linearly independent.

So Lemma \ref{signdet} applies, and conclude that $1, y_1, \ldots,
y_n$ are $K$-linearly independent. As mentioned above, this implies
that $K(x_1,\dots,x_n)^G=F(y_1,\dots,y_n)$.  This finishes the proof
in the transitive case; the proof in the general case follows  as in Theorem~\ref{nonamenonsign}.
\end{proof}


\begin{example}
Let $K= \Q(\rho)$, where $\rho$ is a primitive $5$-th root of unity,
so that $K/\Q$ is Galois, with $\mathrm{Gal}(K/\Q) \cong C_4$.
Take $n=3$, assume $G\leq \mathrm{GL}(3,\Z)$ is generated by $$
\sigma = \begin{bmatrix}
0&-1&0\\
1&0&0\\
0&0&-1
\end{bmatrix}$$ and let $(x_1, x_2, x_3)$ be
indeterminates over $K$, on which $G$ acts as in Definition~\ref{Assumption};
this action is not transitive.
One can also verify that $G$ is isomorphic to $\mathrm{Gal}(K/\Q)$;
through this action, it acts on the $5$-th roots of unity as follows:
\begin{table}[H]
\centering
\begin{tabular}{l|llll} 
& id & $\sigma$ & $\sigma^2$ & $\sigma^3$\\
\hline
%% $x_1$  & $x_1$ & $x_2^{-1}$ & $x_1^{-1}$ & $x_2$ \\
%% $x_2$ & $x_2$ & $x_1$ & $x_2^{-1}$ & $x_1^{-1}$ \\
%% $x_3$ & $x_3$ & $x_3^{-1}$ & $x_3$ & $x_3^{-1}$ \\
$\rho$ & $\rho$ & $\rho^2$ & $\rho^4$ &  $\rho^3$\\
$\rho^2$ & $\rho^2$ & $\rho^4$ & $\rho^3$ &  $\rho$\\
$\rho^3$ & $\rho^3$ & $\rho$ & $\rho^2$ &  $\rho^4$\\
$\rho^4$ & $\rho^4$ & $\rho^3$ & $\rho$ &  $\rho^2$
\end{tabular}.
\end{table}
We choose \todo{who?} as our normal element of the extension $K/\Q$.

With $(z_1,z_2,z_3)$ and $(y_1,y_2,y_3)$
defined as before, the matrix giving the coordinates of
$(y_1,y_2,y_3)$ on the basis $(1,x_1,x_2,x_3)$ is
$$C= \begin{bmatrix}
\rho^2+\rho^4 & \rho -\rho^4 & \rho^3 -\rho^2 & 0\\
\rho^3+\rho^4 & \rho^2 -\rho^3 & \rho -\rho^4 &0 \\
\rho^2+\rho^3 &0 &0 & \rho - \rho^2 -\rho^3+\rho^4
\end{bmatrix}.
$$ Remark that due to the non-transitivity of the action of $G$, the
right-hand $3 \times 3$ submatrix of $C$, while invertible, does not satisfy
the assumptions of Lemma~\ref{signdet} (this matrix is block diagonal,
with blocks corresponding to $K(x_1,x_2)$ and $K(x_3)$, for which the
lemma applies).
\end{example}


\bibliographystyle{plain}
\bibliography{bibliography}

\end{document}
